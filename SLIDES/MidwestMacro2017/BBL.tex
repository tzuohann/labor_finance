\documentclass[hyperref={pdfpagelabels=false}]{beamer}
\usepackage{graphicx,lmodern,subfigure,ulem,color,graphicx,tikz,booktabs}
\usetheme{default}
%\usecolortheme{seahorse}
\definecolor{beamer@blendedblue}{rgb}{0.1,0.5,0.1}
\definecolor{ForestGreen}{RGB}{60, 140, 60}
\setbeamercolor{structure}{fg=beamer@blendedblue}
\setbeamertemplate{navigation symbols}{}
\setbeamertemplate{footline}[frame number]

\usepackage{lmodern}
\newcommand{\spitem}{\vspace{.3cm}\item}

%commands
\newcommand{\elas}{$E_{labor}$}

\title[Capital Structure]{How labor market frictions affect capital structure\\ } 
\author[\insertframenumber/\inserttotalframenumber \hskip 1in 
]{Yasser Boualam, Marco Brianti, Tzuo Hann Law
}
\institute{UNC, BC, BC}

\date{November 21, 2016}

\begin{document}
\frame{\titlepage \begin{center} Midwest Macro, Pittsburgh, 2017 \end{center} }


\frame{\frametitle{How does labor market frictions affect capital structure?}
\begin{figure}
	\centering
	\begin{itemize}
		\item Modigliani Miller 1958
		\item[] \textbf{Why does capital structure matter at all?}
		\item[] Bankruptcy costs can be high(er) after accounting for stakeholders who might not be (fully) represented at the bargaining table. 
		\item A firm's labor force is one such under-represented entity.
		\item \textbf{This paper:} How does adding capital structure to a workhorse labor market search model affect capital structure decisions?
	\end{itemize}
\end{figure}
}

\frame{\frametitle{What we do}
	\begin{figure}
		\centering
		\begin{itemize}
			\item Highlight empirical findings in the literature that call for the models we present.
			\item Present a simple three period model to highlight the channels.
			\item Present a fully dynamic model and do something...
		\end{itemize}
	\end{figure}
}


\frame{\frametitle{Main channels}
	\begin{figure}
		\centering
		\begin{itemize}
			\item Absent any search frictions, owners of production utilize optimal quantities of debt.
			\item With labor market frictions, the firm partners with a risk averse worker who potentially has the option to quit the partnership.
			\item While this quitting in a partial equilibrium setting benefits workers ex-post, it leads to less entry, less-than-optimal debt use, lower equilibrium wages and ex-ante lower value to workers.
		\end{itemize}
	\end{figure}
}

\frame{\frametitle{Literature}
	\begin{figure}
		\centering
		\begin{itemize}
			\item 
			\item
			\item
		\end{itemize}
	\end{figure}
}

\frame{\frametitle{Empirical observations}
	\begin{figure}
		\centering
		\begin{itemize}
			\item 
			\item
			\item
			\item
		\end{itemize}
	\end{figure}
}

\frame{\frametitle{Model without Labor Market Frictions}
	In this simple setup, there are no search frictions and the firm owner simply considers a debt choice . The firm would be identical to a risk neutral worker.
	\begin{figure}
		\centering
		\begin{itemize}
			\item 
			\item
			\item
			\item
			\item
		\end{itemize}
	\end{figure}
}

\frame{\frametitle{Simple Model with Labor Market Frictions}
	The worker owns the entire production unit. The firm would be identical to a risk neutral worker.
	\begin{figure}
		\centering
		\begin{itemize}
			\item 
			\item
			\item
			\item
			\item
		\end{itemize}
	\end{figure}
}

\frame{\frametitle{Dynamic Model with Labor Market Frictions}
	The worker owns the entire production unit. The firm would be identical to a risk neutral worker.
	\begin{figure}
		\centering
		\begin{itemize}
			\item 
			\item
			\item
			\item
			\item
		\end{itemize}
	\end{figure}
}

\frame{\frametitle{Conclusion}
	\begin{figure}
		\centering
		\begin{itemize}
			\item 
			\item
			\item
			\item
			\item
		\end{itemize}
	\end{figure}
}

\end{document}
