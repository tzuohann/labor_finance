\documentclass[hyperref={pdfpagelabels=false}]{beamer}
\usepackage{graphicx,lmodern,subfigure,ulem,color,graphicx,tikz,booktabs}
\usetheme{default}
%\usecolortheme{seahorse}
\definecolor{beamer@blendedblue}{rgb}{0.1,0.5,0.1}
\definecolor{ForestGreen}{RGB}{60, 140, 60}
\setbeamercolor{structure}{fg=beamer@blendedblue}
\setbeamertemplate{navigation symbols}{}
\setbeamertemplate{footline}[frame number]

\usepackage{lmodern}
\newcommand{\spitem}{\vspace{.3cm}\item}

%commands
\newcommand{\elas}{$E_{labor}$}

\title[Capital Structure]{How labor market frictions affect capital structure\\ } 
\author[\insertframenumber/\inserttotalframenumber \hskip 1in 
]{Yasser Boualam, Marco Brianti, Tzuo Hann Law
}
\institute{UNC, BC, BC}

\date{\today}

\begin{document}
\frame{\titlepage \begin{center} Midwest Macro, Pittsburgh, 2017 \end{center} }


\frame{\frametitle{How does labor market frictions affect capital structure?}
\begin{figure}
	\centering
	\begin{itemize}
		\item Modigliani Miller 1958
		\item[] \textbf{Why does capital structure matter at all?}
		\item[] Bankruptcy costs can be high(er) after accounting for stakeholders who might not be (fully) represented at the bargaining table. 
		\item A firm's labor force is one such under-represented entity.
		\item \textbf{This paper:} How does adding capital structure to a workhorse labor market search model affect capital structure decisions?
	\end{itemize}
\end{figure}
}

\frame{\frametitle{What we do}
	\begin{figure}
		\centering
		\begin{itemize}
			\item Highlight empirical findings in the literature that call for the models we present.
			\item Present a simple three period model to highlight the channels.
			\item Present a fully dynamic model and do something...
		\end{itemize}
	\end{figure}
}


\frame{\frametitle{Main channels}
	\begin{figure}
		\centering
		\begin{itemize}
			\item Absent any search frictions, owners of production utilize optimal quantities of debt.
			\item With labor market frictions, the firm partners with a risk averse worker who potentially has the option to quit the partnership.
			\item While this quitting in a partial equilibrium setting benefits workers ex-post, it leads to less entry, less-than-optimal debt use, lower equilibrium wages and ex-ante lower value to workers.
		\end{itemize}
	\end{figure}
}

\frame{\frametitle{Literature}
	\begin{figure}
		\centering
		\begin{itemize}
			\item 
			\item
			\item
		\end{itemize}
	\end{figure}
}

\frame{\frametitle{Empirical observations}
	\begin{figure}
		\centering
		\begin{itemize} 
			\item
			\item
			\item
		\end{itemize}
	\end{figure}
}

\frame{\frametitle{Model without Labor Market Frictions}
	\begin{figure}
		\centering
		\begin{itemize}
			\item Debt is riskless. Borrows pay interest rate $ r $ and return all borrowed capital.
			\item A single agent with initial wealth chooses debt to maximize payoffs in two periods. The output in the first period must be weakly positive.
			\item[] \[\max_{D} \mathbb{E}u(c_1) + \beta \mathbb{E}u(c_2)\]
			\item where \[c_t(\phi_t) = \phi_t(W + D)^\gamma - rD\] is some decreasing returns production function. Productivity shock $ \phi_t \in U[0,1]$ and $ c_2  = b $ for sure if $ c_1 < 0 $.
		\end{itemize}
	\end{figure}
}

\frame{\frametitle{Model without Labor Market Frictions: Solution}
	\begin{figure}
		\centering
		\begin{itemize}
			\item In this setup the optimal choice for debt $ D $ is defined by
			\[ads\] 
			where the trade-off is between producing a positive quantity in the second period in order to obtain a chance at producing in the last period where the minimum level of production is $ b $.
		\end{itemize}
	\end{figure}
}

\frame{\frametitle{Model without Labor Market Frictions: Solution}
	\begin{figure}
		\centering
		\begin{itemize}
			\item The first order condition from earlier yields
			\[ads\] 
			where we how the incompleteness of markets drives a wedge in the typical solution for equation the expected return of capital to the interest rate $ r $.
			\item Finally, note here that the owner of the firm can be the worker or the firm in a setting with both agents. 
		\end{itemize}
	\end{figure}
}


\frame{\frametitle{Labor Market Frictions with Capital Structure}
	Next, we consider how labor market frictions affects debt choice.
	\begin{figure}
		\centering
		\begin{itemize}
			\item Mortensen and Pissarides style search frictions.  
			\item Entrepreneurs/firms own wealth $ W $ and borrow at rate $ r $. Debt is riskless.
			\item Debt choice is made before entry. No new debt or equity.
			\item Wage contracts are specified by \textit{unconstrained wages}, $ \tilde{w} $.
			\item $ \tilde{w} $ is restricted to be identical in both periods.
			\item Perfect commitment assumed. 
			\item No storage technology. 
		\end{itemize}
	\end{figure}
}

\frame{\frametitle{Timing}
	\begin{figure}
		\centering
		\begin{enumerate}
			\item \textbf{Period 0.} Firms with wealth, $ W $ choose debt $ D $ and enter.
			\begin{itemize}
				\item All workers are unemployed.
				\item Firm's post wage contracts, matching occurs.
				\item Unmatched firms exit immediately.
			\end{itemize}
			\item \textbf{Period 1.} Draw productivity $ \phi_1 $.
			\begin{itemize}
				\item If output is weakly negative, match is broken. Firm exits.
				\item Production + consumption occurs.
				\item Unmatched workers consume $ b $.
			\end{itemize}
			\item \textbf{Period 2.} Draw roductivity $ \phi_2 $.
			\begin{itemize}
				\item Separation if output is below $ b $. 
				\item Production + consumption occurs. 
				\item Unmatched workers consume $ b $.
			\end{itemize}
		\end{enumerate}
	\end{figure}
}

\frame{\frametitle{Period production}
	\begin{figure}
		\centering
		\begin{itemize}
			\item Period output is given by
			\item[] \[\phi_t (W + D)^\gamma - Dr\]
			\item If period output is negative, exit occurs.
			\item If output exceeds $ \tilde{w} $, workers are paid $ \tilde{w} $. 
			\item Dividends are positive iff $ (W + D)^\gamma - Dr \ge \tilde{w} $
			\item Don't worry, we have pictures.
		\end{itemize}
	\end{figure}	
}

\frame{\frametitle{Period 1 Wages}
\begin{figure}
	\includegraphics[scale=0.3]{figures/WagePeriod1}
	\label{fig:wageperiod1}
\end{figure}
}

\frame{\frametitle{Period 2 Wages}
	\begin{figure}
		\includegraphics[scale=0.3]{figures/WagePeriod2}
		\label{fig:wageperiod1}
	\end{figure}
}




\frame{\frametitle{Promised Value of a Contract}
	\begin{figure}
	\centering
	\begin{itemize}
		\item $ E(\tilde{w}) $ is the promised value of contract $ \tilde{w} $.
\begin{eqnarray*}
		E(\tilde{w}) &=& \underbrace{\phi_e (1+\beta) u(b)}_\text{$ f(\phi_1) < 0 $, exit} \\
					 &+& \underbrace{\int_{\phi_e}^{\phi_{dw}} f(\phi_t) d\phi}_\text{wage = output, zero div.} + \underbrace{\int_{\phi_{dw}}^1\tilde{w}d\phi}_\text{wage = $ \tilde{w} $, positive div.} \\
					 &+& (1-\phi_e)\underbrace{\left( \phi_b u(b) + \int_{\phi_b}^{\phi_{dw}} f(\phi_t) d\phi + \int_{\phi_{dw}}^1\tilde{w}d\phi\right)}_\text{final period wages} 
		\end{eqnarray*}
		\item[] where $ \phi_e $, $ \phi_b $ and $ \phi_dw $ are the cutoffs seen earlier.
	\end{itemize}
\end{figure}	
}

\frame{\frametitle{Worker's Problem}
	\begin{figure}
		\centering
		\begin{itemize}
		\item $ \theta(\tilde{w}) $ is market tightness for a given contract
		\item $ p(\theta(\tilde{w})) = m(\theta(\tilde{w}))/s $ is job finding probability
		\item \[U = \max_{\tilde{w}} \underbrace{p(\theta(\tilde{w})) E(\tilde{w})}_\text{indifference condition} \]
	\end{itemize}
\end{figure}	
}

\frame{\frametitle{Expected Profits of a Contract}
	\begin{figure}
		\centering
		\begin{itemize}
			\item $ V(\tilde{w}) $ is the value of contract $ \tilde{w} $ taking debt as given
			\begin{eqnarray*}
				V(\tilde{w}) &=& \underbrace{\phi_e (1+\beta) \cdot 0}_\text{$ f(\phi_1) < 0 $, exit} \\
				&+& \underbrace{\int_{\phi_e}^{\phi_{dw}} 0 \hspace{1mm} d\phi}_\text{wage = output, zero div.} + \underbrace{\int_{\phi_{dw}}^1f(\phi_1)-\tilde{w} \hspace{1mm}d\phi}_\text{wage = $ \tilde{w} $, positive div.} \\
				&+& (1-\phi_e)\underbrace{\left( \phi_b \cdot 0 + \int_{\phi_b}^{\phi_{dw}} 0 \hspace{1mm} d\phi + \int_{\phi_{dw}}^1f(\phi_2)-\tilde{w} \hspace{1mm}d\phi\right)}_\text{final period wages} 
			\end{eqnarray*}
			\item[] where $ \phi_e $, $ \phi_b $ and $ \phi_dw $ are the cutoffs seen earlier.
		\end{itemize}
	\end{figure}	
}

\frame{\frametitle{Firms's Problem}
	\begin{figure}
		\centering
		\begin{itemize}
			\item $ q(\theta(\tilde{w})) = m(\theta(\tilde{w}))/v $ is vacancy filling probability
			\item[] \[W = \max_{\tilde{w};D} \underbrace{q(\theta(\tilde{w})) V(\tilde{w};D)}_\text{indifference condition} \]
			\item Optimal debt choice will involve firms choosing debt and posting the corresponding profit maximizing contract $ \tilde{w} $ which maximizes ex-ante value, $ U $ for workers. 
		\end{itemize}
	\end{figure}	
}


\frame{\frametitle{Results: Wages}
	\begin{figure}
		\centering
		\begin{itemize}
			\item
		\end{itemize}
	\end{figure}	
}

\frame{\frametitle{Results: Entry}
	\begin{figure}
		\centering
		\begin{itemize}
			\item
		\end{itemize}
	\end{figure}	
}

\frame{\frametitle{Results: Ex-ante Value of Unemployment}
	\begin{figure}
		\centering
		\begin{itemize}
			\item
		\end{itemize}
	\end{figure}	
}

\frame{\frametitle{Results: Profits condition on Matching}
	\begin{figure}
		\centering
		\begin{itemize}
			\item
		\end{itemize}
	\end{figure}	
}

\frame{\frametitle{Dynamic Model with Labor Market Frictions}
	\begin{figure}
		\centering
		\begin{itemize}
			\item 
			\item
			\item
			\item
			\item
		\end{itemize}
	\end{figure}
}

\frame{\frametitle{Conclusion}
	\begin{figure}
		\centering
		\begin{itemize}
			\item 
			\item
			\item
			\item
			\item
		\end{itemize}
	\end{figure}
}

\end{document}
